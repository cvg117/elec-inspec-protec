\documentclass{article}
% \usepackage[utf8]{inputenc} % allow utf-8 input
% \usepackage[T1]{fontenc}    % use 8-bit T1 fonts
% \usepackage{hyperref}       % hyperlinks
% \usepackage{url}            % simple URL typesetting
% \usepackage{booktabs}       % professional-quality tables
% \usepackage{amsfonts}       % blackboard math symbols
% \usepackage{nicefrac}       % compact symbols for 1/2, etc.
% \usepackage{microtype}      % microtypography
% \usepackage{xcolor}         % colors
% \usepackage{environ}
% \usepackage{neurips_2022}

\title{Redistricting Ideology}


\author{
  Drew Keller \\
  Alejandro Navarrete \\
  Victor Perez Martin \\
  Evelyn Siu \\
  Cole von Glahn \\
  The University of Chicago
  Computational Analysis and Public Policy
}


\begin{document}


\maketitle


\begin{abstract}
  Partisan redrawing of voting maps in the United States is a central strategy for collecting and 
  maintaining political power. Advances in algorithmic tools for this purpose foreshadow a 
  deepening and hardening of these ideological bastions. Preparing responses to this potential
  future requires an understanding of the relationship between voter and representative ideology.
  Linking precinct demographics to candidate ideologies through vote share provides a proxy for the 
  ideological spectrum at the precinct level. Feeding this information through predictive models
  predicated on potetial redistricted maps expresses the expected ideological score of a potential
  district, and therefore the anticipated ideological position of the "most appropriate" representative.
  Analyzing the delta between these predictions and future outcomes provides the public with a window 
  into the extent to which their beliefs and interests are being accurately represented by their
  Congressperson.
\end{abstract}


\section{Datasets}


Our work leverages data from the following four datasets. Our models will be
tested on Michigan's voting maps and political information. It will be trained
on the political information from Pennsylvania, Wisconsin, and Minnesota which
share many demographic, regional, and political features with Michigan. 

\subsection{Candidate Ideology Scoring}


Stanford's Database on Ideology, Money, and Elections (DIME) includes
Candidate ideology scores. These scores are assigned based on
the ideological positions of the Candidate's donors, as well as those
causes/candidates to which the Candidate donates. Scores are provided
for Candidates who reach the General Election.


\subsection{Precinct-Level Voting Results}


Harvard's dataverse provides access to the precinct-level vote share for 
all candidates in the General Elections for Congressional seats. 


\subsection{Voting Maps}


Michigan's public redistricting process provides access to a number of
proposed political maps. We use the current map as a control, and expand
to the proposed maps providing varied precinct-to-district
configurations for analysis.


\subsection{Demographic Data}


We use the American Community Survey's 5-year estimates on demographic data.
As redistricting is often focused on voters' demographic qualities, the 
intersection of demography and ideology is a window to the strategic decisions
being made by candidates and party bodies as they construct political geography.


\section{Model Details}


At first stage we will be constructing two simple models: a decision tree
and a logistic regression. 

The decision tree is intended to simulate a simplified strategic 
decision-making process, wherein the tree acts as a redistricting 
committee splitting on demographic features. It will label potential
districts with ideological leans based on predictions made by
categorizing their component precincts.

The logistic regression is intended to provide a deeper understanding
of the underlying probabilities that certain features result in 
particular ideological skews. At the first stage, it serves as a
backup to ensure that our probability measurements are working.


\section{Graphs and Charts}


NONE AT THIS TIME


\section{Theoretical Results and Assumptions}

FILL IN WITH NEW REPORTS


\section{Limitations}

We expect our analysis to be limited by the following factors:

\begin{itemize}

\item The DIME dataset is missing a small number of candidates

\item The DIME dataset is a back-out of ideology, relying on 3rd
party contributors as measurement. It has the advantage of
representing more candidates who received votes, thereby increasing
our ability to understand the granular aspects of precinct voting
preferences. However, it may be a less precise metric than scoring
systems that narrow their scope to election winners and label
politicians based on their voting records.

\item The census datasets do not label their geographic units 
in the same way as our other datasets. For this round we defaulted
to county level measurement. We are continuing to develop our 
matching strategy at the precinct level.
\end{itemize}

\section{Negative Societal Impacts}

There is the potential that this work is used by political operatives to
further understand and abuse differences in ideology and demography. Rather
than a tool for changing candidate behavior and encouraging a closer match of
representative to represented, it may be leveraged to instead alter the playing
field in favor of particular candidates and their ideological beliefs.

\section{Asset Attribution}


Bonica, Adam. 2019. Database on Ideology, Money in Politics, and Elections: Public version
3.0 [Computer file]. Stanford, CA: Stanford University Libraries. <http://data.stanford.edu/
dime>.


Bonica, Adam. 2018. “Inferring Roll-Call Scores from Campaign Contributions Using Super-
vised Machine Learning” American Journal of Political Science, 62, (4): 830-848. (https://
onlinelibrary.wiley.com/doi/full/10.1111/ajps.12376).

U.S. Census Bureau. (2012). 2020 American Community Survey.  
Retrieved from https://www.census.gov/data/datasets.html.

Voting and Election Science Team, 2019, "2018 Precinct-Level Election Results", 
https://doi.org/10.7910/DVN/UBKYRU, Harvard Dataverse, V59.


\section{Conclusion}

FILL THIS IN

\begin{itemize}
CHANGE THIS IF HELPFUL
\end{itemize}

\end{document}